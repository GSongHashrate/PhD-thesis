\addcontentsline{toc}{chapter}{Appendices}

% The \appendix command resets the chapter counter, and changes the chapter numbering scheme to capital letters.
%\chapter{Appendices}
\appendix
\chapter{Full Instruction for ElimLin Experiments}
\label{appendixlabel1}

Software Setup:
\begin{enumerate}
	\item \textbf{Simon.exe}: http://www.nicolascourtois.com/software/simon.exe \\ 
	- Source code: https://github.com/GSongHashrate/SimonSpeck \\
	- Documentation: section Simon inside: http://www.cryptosystem.net/aes/toyciphers.html
	\item \textbf{ax64.exe}:  http://www.nicolascourtois.com/software/ax64.exe \\
	- Documentation: http://www.cryptosystem.net/aes/tools.html
	\item \textbf{cryptominisat-2.9.6-win64.exe} http://www.nicolascourtois.com/software/cryptominisat-2.9.6.-win64.exe \\
	-  Sources and documentation: http://www.msoos.org/cryptominisat2/
\end{enumerate}

\textbf{Command Line}:\\
Simon.exe NR /fixkF /insK /xl0, where NR = number of rounds for the cipher, F = number of fixed/guessed bits for the key and K = number of random plaintext/ciphertext pairs used.

Command Line output might look like: \\
\noindent\fbox{%
	\parbox{\textwidth}{%
		Simon.exe 16 /fixk0 /ins8 /xl0\\
		... \\
		10496+ 512+ 448+ 257+ 1+ 0 \\
		Elim[ 14208] 0.529 h 2494/3712 ... 
	}%
}

In this case NR = 16, F = 0, K = 8. The output is interpreted as follows:\\
\noindent\fbox{%
	\parbox{\textwidth}{%
Simon.exe 16 /fixk0 /ins8 /xl0 \\
... \\
$r_0$+ $r_1$+ $r_2$+ $r_3$+ $r_4$+ $r_5$ \\
Elim[ $Total (=r_0+V_{start})$] 0.529 h $V_{Unbroken}$/$V_{start}$ ...
}%
}

\begin{table}[h!]
	\centering
	\caption{Data gathered by running ElimLin on 7 rounds of Simon 64/128}
	\label{tbl:fullResultsSecrypt20167R}
	\begin{tabular}{lllllllllll}
		NR & K & $V_{start}$ & $V_{unbroken}$ & $r_1$   & $r_2$  & $r_3$  & $r_4$ & $r_5$ & $r_6$ & $r_7$ \\ \hline
		7  & 2 & 448    & 248       & 1472 & 128 & 64  & 8  & 0  &    &    \\
		7  & 3 & 608    & 268       & 2208 & 192 & 128 & 20 & 0  &    &    \\
		7  & 4 & 768    & 271       & 2944 & 256 & 192 & 49 & 0  &    &   \\
		7  & 5 & 928    & 262       & 3680 & 320 & 256  & 89  & 1  &   0 &    \\
		7  & 6 & 1088    & 244       & 4416 & 384 & 320 & 139 & 1  &   0 &    \\
		7  & 7 & 1248   & 215       & 5152 & 448 & 384 & 200 & 1  &   0 &   \\
		7  & 8 & 1408    & 188       & 5888 & 512 & 448  & 259  & 1 &   0 &    \\
		7  & 9 & 1568    & 158       & 6624 & 576 & 512 & 320 & 2  &  0  &    \\
		7  & 10 & 1728    & 126       & 7360 & 640 & 576 & 384 & 2  & 0   &   \\
		7  & 11 & 1888    & 82       & 8096 & 704 & 640  & 448  & 14  &  0  &    \\
		7  & 12 & 2048    & 0       & 8832 & 768 & 704 & 512 & 17  & 47   & 0    
								
	\end{tabular}
\end{table}

\begin{table}[h!]
	\centering
	\caption{Data gathered by running ElimLin on 8 rounds of Simon 64/128}
	\label{tbl:fullResultsSecrypt20168R}
	\begin{tabular}{lllllllllll}
		NR & K & $V_{start}$ & $V_{unbroken}$ & $r_1$   & $r_2$  & $r_3$  & $r_4$ & $r_5$ & $r_6$ & $r_7$ \\ \hline
		8  & 2 & 512    & 317       & 1600 & 128 & 64  & 3  & 0  &    &    \\
		8  & 4 & 896    & 409       & 3200 & 256 & 192 & 39 & 0  &    &    \\
		8  & 8 & 1664    & 443       & 6400 & 512 & 448 & 261 & 0  &    &   \\
		8 & 10 & 2048    & 448       & 8000 & 640 & 576  & 384  & 0  &    &    \\
		8 & 12 & 2432    & 448       & 9600 & 768 & 704 & 512 & 0  &    &    \\
		8  & 16 & 3200   & 444       & 12800 & 1024 & 960 & 768 & 4  &   0 &   \\
		8  & 20 & 3968    & 439       & 16000 & 1280 & 1216  & 1024  & 9 &   0 &    \\
		8 & 25 & 4928    & 435       & 20000 & 1600 & 1536 & 1344 & 13  &  0  &    \\
		8  & 30 & 5888    & 525       & 24000 & 1920 & 1856 & 1664 & 23  & 0   &   \\
		8 & 32 & 6272    & 417       & 25600 & 2048 & 1984 & 1792 & 31  &  0  &    \\
		8  & 40 & 7808    & 402       & 32000 & 2560 & 2496 & 2304 & 46  & 0   &   \\
		8 & 50 & 7808    & 378       & 40000 & 3200 & 3136 & 2944 & 70  &  0  &    \\
		8  & 64 & 12416    & 307       & 51200 & 4096 & 4032  & 3840  & 140  &  1  & 0   \\
		8  & 70 & 13568    & 249       & 56000 & 4480 & 4416 & 4224 & 189  & 10   & 0    
		
	\end{tabular}
\end{table}
\chapter{Java Tool for Deep Inspection of ElimLin} \label{deepLin}
Guangyan Song and Nicolas Courtois:
``Java tool for DEEP INSPECTION of equations generated with ElimLin over GF(2)
in Cryptalanalysis of Block Ciphers'',
available at
\url{http://www.nicolascourtois.com/software/DeepElimlin-1.4-SNAPSHOT.jar}. Documentation can be found in the appropriate section of this web page:
\url{http://www.cryptosystem.net/aes/tools.html}.

\chapter{Examples of Cracked Brainwallet Passwords}
\label{appendixlabel2}
We have found over 18,000 cracked Brainwallet passwords. In 2016, our open source tool was given to students for a UCL code breaking competition in module GA18. More than 100 new passwords were found by MSc students: Iason Papapanagiotakis, Jeonghyuk Park, Ellery Smith, Weixiu Tan and Wei Shao. Here we only list some interesting passwords. The full result remains confidential for security reasons. 
\begin{enumerate}
	\item say hello to my little friend
	\item to be or not to be
	\item Walk Into This Room
	\item party like it's 1999
	\item yohohoandabottleofrum
	\item dudewheresmycar
	\item dajiahao
	\item hankou
	\item \{1summer2leo3phoebe
	\item 0racle9i
	\item andreas antonopoulos
	\item Arnold Schwarzenegger
	\item blablablablablablabla
	\item for the longest time
	\item captain spaulding
\end{enumerate}


 % description of document, e.g. type faces, TeX used, TeXmaker, packages and things used for figures. Like a computational details section.
% e.g. http://tex.stackexchange.com/questions/63468/what-is-best-way-to-mention-that-a-document-has-been-typeset-with-tex#63503

% Side note:
%http://tex.stackexchange.com/questions/1319/showcase-of-beautiful-typography-done-in-tex-friends