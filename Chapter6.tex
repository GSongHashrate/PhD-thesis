\chapter{Algebraic Cryptanalysis of Simon}\label{ch:SIMON}
In Chapter \ref{ch:GOST} we have introduced the notion of contradiction immunity and SAT immunity. We have shown how an optimized guess-then-solve attack can be done on GOST. In this chapter, we are going to explore how to improve algebraic attack with well chosen data. Our target is the new NSA block cipher Simon \cite{NSAciphers}. The lightweight block cipher Simon was introduced by NSA in 2013 and later submitted to ISO to become an international standard. Similar to GOST, Simon has very low diffusion and small S-boxes, which makes it an excellent target for algebraic cryptanalysis. Our aim is to use the very rich algebraic structure
with additional data provided ( e.g. pairs $\{(P,P'),(C,C')\}$ which
follow a certain highly probable truncated differential property) in order to solve
the underlying multivariate system of equations. We attempt to solve the system
by either using a SAT solver (after converting the system to its corresponding
CNF-SAT form with the Courtois-Bard-Jefferson method \cite{BardCourtoiJeffersonConv}) or by the ElimLin algorithm \cite{FourMNL,OptimiPaper,OptimiPaper2}.
We are able to break up to 10 (/44) rounds of the cipher using
a SAT solver and the usual guess-then-determine techniques.
Surprisingly, in most cases we are
able to obtain the key without guessing any key bits
when truncated differentials are used.
This is a very remarkable result as it gives a good hope that the attack will scale up well for a larger number of rounds. This is possibly due to the very low non-linearity of the cipher and suggests that it is worth studying a specific strategy for P/C pairs which have a certain structure and decrease even more the non-linearity of the system by introducing
more linear equations (e.g. truncated differential properties)
until the key can be obtained even for a larger number of rounds. We will discuss in details our results in Section \ref{Sec:SimonResult}.


\section{How to Write Simon Equations}
In Figure \ref{fig:ch3ACmodeling} we showed a general example of modelling a block cipher into MQ equation systems. Here we give a concrete example for writing Simon64/128 in a single round.

Recall the Simon structure in Figure \ref{fig:SIMONroundfn}, let $ek_i$ be the $i$-th key bit used in $Nr$-th round of Simon64/128 encryption. The 32-bit key has a binary representation: $ek=\left( ek_{31},ek_{30}...ek_{0}\right) $. Let $ZR^{Nr}_i$ and $ZL^{Nr}_i$ be the $i$-th bit of $Nr$-th round right and left side input, and similarly $ZR^{Nr+1}_i$, $ZL^{Nr+1}_i$ for the $i$-th bit of output. Then we have:

\begin{center}
$ek_0+ZL^{Nr}_{30} + ZR^{Nr}_0+ZL^{Nr+1}_0+ZL^{Nr}_{24}*ZL^{Nr}_{31} = 0$\\
$ZR^{Nr+1}_0+ZL^{Nr}_0 = 0$ \\
$ek_1+ZL^{Nr}_{31} + ZR^{Nr}_1+ZL^{Nr+1}_1+ZL^{Nr}_{25}*ZL^{Nr}_{0} = 0$\\
$ZR^{Nr+1}_1+ZL^{Nr}_1 = 0$ \\
$\dots$ \\
$ek_{31}+ZL^{Nr}_{29} + ZR^{Nr}_{31}+ZL^{Nr+1}_{31}+ZL^{Nr}_{23}*ZL^{Nr}_{30} = 0$\\
$ZR^{Nr+1}_{31}+ZL^{Nr}_{31} = 0$ 
\end{center}

Similarly, one can simply write equations for the key extension function, cf. \cite{NSAciphers}. More details can be found in our Simon implementation code (including equation generator for all versions) which is available online \cite{SongSimonSpeckGithub}.
\section{Differential-Algebraic Cryptanalysis of Simon}\label{Sec:DAcSimon}

We evaluated the security of Simon against algebraic attacks under the following
three settings (cf. Fig. \ref{ThreeScenarios}), where S=Similar and R=Random
as explained in the introduction.

\begin{itemize}
	\item \textbf{Setting 1:}  Known Plaintext Attack. Random P/C pairs are available (RP/RC). 
	\item \textbf{Setting 2:} One type of Chosen Plaintext Attack.  Random P/C pairs are available with plaintexts of low Hamming distance (SP/RC).
	\item \textbf{Setting 3:} Some form of Chosen Plaintext and Ciphertext Attack. Random P/C pairs which satisfy a truncated differential property in the input and output of the reduced-version of the cipher we study.  (SP/SC) %More random P/C pairs are generated and used. 
	% \item \textbf{Setting 3':}(SP/SC)' Random P/C pairs which satisfy a truncated differential property in the input and output of the reduced-version of the cipher we study. Then we generate more such pairs by using pairs with similar plaintexts (low Hamming Distance)
	%    such that ciphertext pairs lie in the truncated differential mask.
\end{itemize}

\begin{figure}[!h]
	\vspace{-0.2cm}
	\centering
	%{\epsfig{file = three_scenarios.eps, width = 7.5cm}}
	\includegraphics*[width=60mm]{./pics/three_scenarios2.jpg}
	\caption{Our three attack scenarios}
	\label{ThreeScenarios}
	\vspace{-0.1cm}
\end{figure}



%\begin{figure}[h!]
%    \centering
%    \includegraphics{three_scenarios.eps}
%    \caption[width=0.2]{Our three attack scenarios}
%    \label{setting}
%\end{figure}

Setting 1 is the simplest setting for understanding how many rounds of Simon
can be broken by simple techniques, assuming availability of
a few P/C pairs. This setting helps us to understand the maximum number of rounds we
can break by guessing as few key bits as possible and using as few  P/C
pairs as possible. It is a non-trivial step in order to set the benchmark for attacking larger number of rounds.


Setting 2 is a form of known-plaintext attack. Setting 2 requires the existence of P/C pairs with plaintexts
of low Hamming distance (or similar plaintexts)
such that many variables are eliminated in the first few rounds due to weak
diffusion. By eliminating some variables from the initial equations we expect that
the system will be solved faster using any solving technique. This is a form of
chosen plaintext attack.

Lastly, Setting 3 assumes the existence of P/C pairs

\begin{center}
	$\{(P_1,C_1),(P_2,C_2),...,(P_n,C_n)\}$
\end{center}

such that
$P_i\oplus P_j \in \Delta P$ and $C_i\oplus C_j \in \Delta C$, for
all $1\leq i,j \leq n$ and some truncated differential masks $\Delta P,\Delta C$
of low Hamming weight which holds with comparatively high probability.
In our attacks we always use 2 pairs which satisfy a given truncated
differential property and then more P/C pairs are generated by using the
first 2 plaintexts and computing the encryptions of new plaintexts which have small Hamming distance from the
first ones. 

The difference
from Setting 2 is that in this case we also eliminate variables
from the last rounds of the cipher, expecting that the system is even easier to solve. In this
attack, we assume (to make it simple) that the entire codebook is available and thus the
data complexity is $2^{64}$.

In all of our attacks we start with an 8-round truncated differential property (see Section \ref{sec:differentialCry}) $\Delta=[00000222 00000080]$
with 4 active bits found by heuristic method:

\begin{equation}
Prob(\Delta\rightarrow \Delta)\simeq2^{-20.51}
\end{equation}

The mask $[00000222 00000080]$ denotes the set of $2^4-1$ possible differences,
excluding the zero difference.
Our detailed results and discussion will be presented in the following section.

\section{Algebraic Attacks experiments and results} \label{Sec:SimonResult}
We run experiments using SAT solvers and ElimLin Algorithm on a machine with the
following characteristics
\begin{small}
\begin{itemize}
	\item CPU: Intel i7-3520m 2.9GHz
	\item RAM: 4G
	\item OS: 64-bit Windows 8
\end{itemize}
\end{small}
Our open source implementation of Simon also includes an equation generator which generates an algebraic equitation system for n round Simon encryption. Once the equation is generated, we use Nicolas Courtois' software \cite{EquationSolving} to either solve the system by ElimLin or convert to CNF file and then solve by a SAT solver. For the reader to check and verify our results, here are the command lines we used to get our experimental results, software are available online \cite{simonref,EquationSolving} :
\begin{small}
\begin{itemize}
	\item Equation generation: Simon.exe NumberOfRounds fixedKey  
	\item ElimLin: xl0.exe /deg2 fileName 
	\item SAT solver: xl0.exe /deg2 /sat /bard /timeout600 /cryptominisat64296 fileName
\end{itemize}
\end{small}
\subsection{Experiments with 2 P/C pairs}
The initial benchmark experiments were done with only 2 P/C pairs and solved by a SAT solver using 8 round Truncated differential mask $[00000222 00000080]$ $\Rightarrow$ $[00000222 00000080]$. Table \ref{tab:example1} presents our 2 P/C pairs experiment results. The average time (in seconds) taken $T_{average}$
to solve the underlying problem
by a SAT solver is presented, while the time complexity $C_T$
(in terms of Simon encryptions) is
computed by the following formulae,

\begin{equation} \label{EQ:time_comp}
C_T=2^{k}\times2^{log_2(T_{average}*N_{8REnc})},
\end{equation}

where $k$ is the number of bits we guess initially, $N_{nREnc}$ is the number of n round Simon encryptions the experiment PC can run in 1 second.

\begin{table}[!hh]
	\caption[Best results obtained by a SAT solver using 2P/C pairs]{Best results obtained by a SAT solver using 2P/C pairs for 8 rounds of Simon64/128. $k$ is the total number of randomly guessed key bits. The time complexity for guessing is $2^{k}$. $\#$(P/C) shows the number of plaintext-ciphertext pairs. $T_{average}$ is the average solving time using a solver. $C_T$  is the total time complexity for breaking the cipher, calculated based on equation \ref{EQ:time_comp}. For settings see detail in figure \ref{ThreeScenarios}}\label{tab:example1} \centering
	\begin{tabular}{|c|c|c|c|c|c|}
		\hline
		$\#$(Rounds) & $k$ & $\#$(P/C) & $T_{average}$(s) & $C_T$ & Setting \\
		\hline
		8 & 84 & 2 & 63.76   & $2^{110.08}$ & RP/RC \\
		8 & 80 & 2 & 87.38   & $2^{106.53}$ & RP/RC \\
		\hline
		8 & 75 & 2 & 156.60  & $2^{102.37}$ & SP/RC \\
		\hline
		8 & 75 & 2 & 515.60  & $2^{104.09}$ & SP/SC \\
		\hline
	\end{tabular}
\end{table}

We start from setting 1 using random Plaintext and random Ciphertext pair (RP/RC), until we cannot solve the equations using SAT solver. We successfully break 8 rounds Simon with guessing some key bits. The best result for 8 rounds is fixing 80 key bits with a complexity of $2^{106.53}$. Fixing less than 80 key bits will not be solved by SAT solver within 600 seconds. Then we try setting 2 and setting 3 to compare the results. Setting 2 using selected Plaintext and Random Ciphertext (SP/RC) and setting 3 using selected Plaintext and selected ciphertext (RP/RC) are better than RP/RC. However, the results in Table \ref{tab:example1} show that SP/SC is not performing any better than SP/RC. Both SP/RC and SP/SC are not improving RP/RC a lot. 

The experiment results show that for setting 2 and setting 3, using only 2 P/C pairs cannot provide enough information (additional linear equations) to perform a better attack than RP/RC. In the next subsection we try to increase the number of P/C pairs and compare the results using different settings.

\subsection{Experiments with more P/C pairs}
We start by using more P/C pairs for 8 rounds Simon and we show our results in Table \ref{tab:8RmorePC}. Here we start to record some features of the converted CNF files. In Table \ref{tab:8RmorePC} and other SAT solver results table, $n,s,h$ stand for number of variables, average sparsity (the average number of literals in each clause) and hardness respectively and $m$ is the number of clauses in the converted CNF file.
We define hardness as a number $h$ such that $h^n$ is the running time, where $n$ is the number of variables.
It is known that $h\leq 1.47$ for 4-SAT problems \cite{semaev}
% explain hardness

\begin{table}[!hh]
	\caption[Best results obtained by a SAT solver for 8 rounds with 6 P/C pairs]{Best results obtained by a SAT solver for 8 rounds with 6 P/C pairs. $n,s,h$ stand for number of variables, average sparsity (the average number of literals in each clause) and hardness respectively and $m$ is the number of clauses in the converted CNF file.
		We define hardness as a number $h$ such that $h^n$ is the running time, where $n$ is the number of variables. }\label{tab:8RmorePC} \centering
	\begin{tabular}{|c|c|c|c|c|c|c|c|c|c|}
		\hline
		$\#$(Rounds) & $k$ & $\#$(P/C) & $T_{average}$(s) & $C_T$ & Setting & $n$ & $x=\frac{m}{n}$ & $s$ & $h$ \\
		\hline
		8 & 45 & 6 & 207.31   & $2^{27.78}$ & RP/RC & 8576	& 6.5118	& 4.2761	& 1.0032 \\
		\hline
		8 & 0  & 6 & 12.21   & $2^{23.76}$ & SP/RC  & 8576	& 6.5065	& 4.2787	& 1.0029  \\
		\hline
		8 & 0  & 6 & 11.84   & $2^{23.57}$ & SP/SC  & 8576	& 6.5513	& 4.2631	& 1.0028 \\
		\hline
	\end{tabular}
\end{table}

%From Table \ref{tab:8RmorePC} we can see that when using more P/C pairs, all the three settings get better results than using 2 P/C pairs in Table \ref{tab:example1}. The improvements for setting 2 and setting 3 are much more larger than setting 1. Setting 2 and setting 3 can break 8R Simon without guessing any key bits.

Then we start to extended the current 8 rounds attack to 9 and 10 rounds. We first extend our 8R truncated differential mask to 9 and 10 rounds as the following:
\begin{itemize}
	\item 9R: $[00000022 00000080]$ $\rightarrow$ $[00022e4c 00000222]$
	\item 10R: $[00000022 00000080]$ $\rightarrow$ $[002eff9a 00022e4c]$ 
\end{itemize}

We manage to break 9 rounds without guessing any key bits and break 10 rounds with some key bits guessing. Our best results are shown in Table \ref{tab:10RmorePC}

\begin{table}[!hh]
	\caption[Best results obtained by a SAT solver]{Best results obtained by a SAT solver. Table set up is the same as table \ref{tab:8RmorePC}. The upper part results are using SP/RC P/C pairs, and the time compliexity for breaking 10 rounds Simon64/128 is $2^{118.5}$ while using SP/SC with truncated differencial properities gives $2^{98.79}$. The results show the power of using well selected data samples in algerbic cryptanlysis. }\label{tab:10RmorePC} \centering
	\begin{tabular}{|c|c|c|c|c|c|c|c|c|c|}
		\hline
		$\#$(Rounds) & $k$ & $\#$(P/C) & $T_{average}$(s) & $C_T$ & Setting & $n$ & $x=\frac{m}{n}$ & $s$ & $h$ \\
		\hline
		9 & 0 & 6 & 222.50   & $2^{27.9}$ & SP/RC & 9536	& 6.70	& 4.31	& 1.0029 \\
		9 & 0 & 7 & 94.7   & $2^{26.6}$ & SP/RC & 11104	& 6.70	& 4.31	& 1.0024 \\
		10 & 90 & 8 & 346.0   & $2^{118.5}$ & SP/RC & 13952	& 6.90	& 4.32	& 1.0020 \\
		\hline
		9 & 0  & 6 & 24.24   & $2^{24.7}$ & SP/SC  & 9536	& 6.69	& 4.31	& 1.0026  \\
		9 & 0  & 7 & 18.56   & $2^{24.3}$ & SP/SC  & 11104	& 6.70	& 4.31	& 1.0022  \\
		10 & 70  & 10 & 417.73   & $2^{98.79}$ & SP/SC  & 17408	& 6.88	& 4.31	& 1.0022  \\
		\hline
	\end{tabular}
\end{table}

Moreover, assuming Setting 3 we can break 10 rounds by
guessing 70 bits of the key initially with a time complexity of $2^{98.79}$ encryptions.
Note that in SP/SC setting we always generate two P/C pairs which satisfy the truncated differential property, and the rest pairs are generated at random.

The other two settings have failed to produce good results for 10 rounds in reasonable time
and this reflects the power of using pairs which follow strong truncated differential properties. We conjecture that for a cipher of low non-linearity, there exists a certain amount of pairs which depends on the linear relations in the cipher which can be used to break any round.
\subsection{ElimLin Results}
Table \ref{tab:SimonElimlin} presents the results using the ElimLin algorithm for solving the
underlying system of equations.


\begin{table}[!h]
\caption{Best results obtained by a ElimLin Algorithm}\label{tab:SimonElimlin} \centering
\begin{tabular}{|c|c|c|c|c|c|}
  \hline
  $\#$(Rounds) & $k$ & $\#$(P/C) & $T_{average}$(s) & $C_T$ & Setting \\
  \hline
  8 & 0 & 6 & 824.4 & $2^{29.8}$ & SP/RC \\
 \hline
 8 & 0 & 6 & 583.2 & $2^{29.3}$ & SP/SC \\
  \hline
\end{tabular}
\end{table}

ElimLin exploits the existence of linear equations in order to solve the system. We have been able to break up to 8 rounds in Setting 3 without guessing any key bits
initially. Setting 1 fails, while Setting 2 is much weaker than Setting 3. The best
attack we have obtained is of time complexity $2^{29.3}$ encryptions against 8 rounds of Simon and requires pairs which satisfy
the truncated differential property presented in the previous section extended for 8 rounds.

Adding pairs which follow a strong truncated differential property is equivalent to adding linear equations in the system and this
is exploited by the ElimLin algorithm. An immediate improvement is to use additional intermediate truncated differences.
This will also eliminate variables in intermediate rounds and introduce more linear
equations in the intermediate rounds. We conjecture that ElimLin is more powerful
in cases where a strong characteristic is found.

\section{Conclusion}
In this Chapter, we studied the security of Simon 64/128 cipher against algebraic attacks and algebraic-differential attacks. We have combined two powerful cryptanalytic techniques: truncated differential cryptanalysis and software algebraic
cryptanalysis. To the best of our knowledge we are the first to show
that such a combination is powerful enough to break up to 10 rounds of a
block cipher. We also demostrated using well selected P/C pairs can significately improve software algebraic cryptanalysis results (compare to random selected P/C pairs). Our work was the first algerbic cryptanalysis work on Simon. Today, it is not the best attack. But it is important for the community to notice Simon's low non-linearity and its low security against algebraic cryptanalysis. 






