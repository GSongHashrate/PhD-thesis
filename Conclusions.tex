\chapter{Conclusion}
\label{chapterlabel4}

This thesis mainly focuses on improving algebraic cryptanalysis with software and solvers. Algebraic cryptanalysis is powerful as it requires small quantities of data, but in general the complexity grows quickly as the number of rounds increases. Many mitigations to improve runtimes are studied. 
We explored different types of optimization processes meant to make algebraic cryptanalysis problems transition from ``hard to solve" to ``easier to solve" by a software solver. We applied these optimizations to concrete ciphers and demonstrated the improvements. We aim to contribute to future government standards, such as Simon and other widely used ciphers including new releases of GOST.  We proposed several possible optimizations for algebraic cryptanalysis and experimentally demonstrated the attacks on GOST and Simon, which were submitted to ISO. We explore many powerful enhancements for algebraic attacks, and in one case we show a result which upper bounds can be obtained, and suggest a new method to predict the complexity of future attacks. 
We also propose an optimized attack for Bitcoin brain wallet attack.  

In Chapter \ref{ch:GOST}, we introduced a new notion of contradiction immunity and SAT immunity, which converts a first stage in cryptanalysis of GOST to an optimization problem. Then we implemented a guess-then-solve attack with a well chosen set of guessed bits. This attack later directly improved the current best attack on GOST. Incidentally GOST was rejected by ISO at that time. 

In Chapters \ref{ch:SIMON} we studied NSA block cipher Simon which was introduced in 2013 and submitted to ISO in 2015. 
We introduced a new method that uses well selected P/C pairs which follow a truncated differential property for algebraic cryptanalysis, and demonstrated the improvement on basic algebraic attacks on Simon with an extremely detailed study of what happens inside the attack and a serious improvement which generates more equations directly. Our work breaks 10/48 rounds of Simon64/128 with less than 10 P/C pairs.
We disagree when some researchers believe that Simon should not be studied by academics: 

\begin{quotation}
	``%`SIMON and SPECK should not even be reviewed by anyone in the community, 
	because it dignifies [the designs] and wastes the cycles – the brain cycles – of intelligent people, by going to look at a thing that is produced by a bad actor agency [(the NSA)]." \\ 
\rightline{{--- Jacob Appelbaum, FSE 2015 invited talk \phantom{This ble} }}
\end{quotation}

We propose an opposite view: it is important for the research community to study Simon because it is likely to become an important industry standard in the future. We published the first algebraic cryptanalysis work on Simon in 2014. Today, it is not the best attack. But it is important for the community to notice Simon's low multipicative complexity, low non-linearity and its low security against algebraic cryptanalysis. 

In our research we spent a lot of time on contemplating what happens inside ElimLin algorithm. It contains a rich variety of attacks, for example, various generalizations of cube attacks not previously studied. 
ElimLin is a powerful tool for algebraic cryptanalysis, but with a fundamental limitation on computational complexity. When trying to solve larger number of rounds, the converted problems get much more bigger and ElimLin can not provide an answer within a short time. With a large number of experiments using ElimLin on Simon, we show that precise prediction for ElimLin is possible. We have made progress in both understanding better and extending/enhancing the ElimLin attack. Our discovery method 
of Section \ref{Sec:ElimLinVsApprox} suggests that the same equations can yet be computed a lot more efficiently. 

Finally, we also looked at the widely used cryptography application --  ECDSA in Bitcoin with the secp256k1 curve. Elliptic curve problems themselves are hard algebraic cryptanalysis problems with complex polynomials and sometimes equations which follow the same topology as in a block cipher. Here nobody is yet able to 
propose advanced practical attacks. Another application we studied is Bitcoin. 
It was invented in 2008 and has grown rapidly since 2012, and it's one of the largest ECC practical applications in the world. We studied how some users manage their private keys and the security pitfalls related to this. Bitcoin uses a special elliptic curve secp256k1 which has not been widely used by any previous application, and in this thesis we provide a detailed benchmark of all the major implementations of this curve, and propose an optimized password cracking attack on Bitcoin brain wallets with a slightly unusual ECC  speed optimization. Our work together with other researchers work had made the Bitcoin community aware that brain wallets are extremely insecure.  
% This just dumps some pseudolatin in so you can see some text in place. 

